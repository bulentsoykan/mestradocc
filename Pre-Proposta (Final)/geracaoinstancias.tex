\chapter{Geração das instâncias}
  
  Atualmente existem diversas fontes na qual se podem obter instâncias para problemas de otimização combinatória sendo uma das mais conhecidas a OR-Library \footnote{ pode ser acessado em http://people.brunel.ac.uk/~mastjjb/jeb/info.html} que foi descrito inicialmente descrito em J.E.Beasley \cite{orlibrary} permitindo o acesso a centenas de conjuntos de instâncias a partir da Internet. 
  
Apesar da existência dessas entidades não foi encontrado nenhuma instância que fosse compatível com o problema de construção de trilhos de aeronaves, fazendo-se então necessário a criação de um conjunto de instâncias próprias que além de permitir a conclusão desse presente trabalho ainda servirá como base para futuras propostas.
  
A obtenção de dados foi feita através da seleção manual do conjunto de voos domésticos cobertos pela empresa de transporte aéreo brasileira denominada TAM (http://www.tam.com.br/) que tinham o tempo de partida na segunda feira e se utilizava do equipamento AirBus Industrie A319. A segunda-feira foi identificada como sendo o dia 0 (zero) apenas para permitir sua utilização no algoritmo. Essa instância que foi obtida é composta por X voos e possui uma grande quantidade de ligações entre os Y aeroportos envolvidos tornando o grau de complexidade mais elevado que instâncias com a características hub-and-spoke que é mais comum nas malhas comerciais norte-americanas. Uma malha é considerada como sendo hub-and-spoke quando existe uma grande concentração de vôos em poucos aeroportos como pode ser visto na Figura K.
  
Para se obter um limite inferior dessas instâncias foi feita uma verificação com o algoritmo do Anexo X que permite checar a quantidade mínima de vôos que colidem em uma determinada janela de tempo que é definida pelo atraso máximo permitido. (Pode-se fazer uma formula para explicar esse funcionamento). Essa quantidade é dito como sendo o limite inferior da instância e é garantido que não existe solução com uma melhor quantidade de trilhos que essa sem que nenhum vôo seja excluído.
  
	A TAM tinha disponível nessa época com N aeronaves desse tipo, logo acreditamos que esse é o número de aeronaves que era necessário para atender a todos esses vôos, fazendo com que reduzir essa quantidade de vôos se tornasse um dos objetivos desse trabalho.
  
	Diversas instâncias também foram geradas a partir dessa, variando o número de voos e as características das malhas  com a finalidade de gerar instâncias com um variado grau de complexidade. Essas instâncias podem ser vistas no Anexo N e podem ser solicitadas diretamente com o autor, porém existe a intenção de adicionar esse conjunto de instâncias na OR-Library.
  
	Ainda é necessário a adição de outras instâncias reais para que a validação dos resultados se tornem mais práticos, para isso é necessário a colaboração de empresas de transporte aéreo uma vez que a obtenção desses dados por meios manuais se mostrou demorado e trabalhoso. 
  