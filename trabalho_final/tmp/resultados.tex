  \chapter{Resultados}

%	Para se obter um limite inferior dessas instâncias foi feita uma verificação
%com o algoritmo do Anexo X que permite checar a %quantidade mínima de voos que
%colidem em uma determinada janela de tempo que é definida pelo atraso máximo
%permitido. (Pode-se fazer uma %formula para explicar esse funcionamento). Essa
%quantidade é dito como sendo o limite inferior da instância e é garantido que
%não existe %solução com uma melhor quantidade de trilhos que essa sem que
%nenhum vôo seja excluído.

Todos os algoritmos descritos foram desenvolvidas na linguagem C++ usando o
solver CPLEX Academic 12 da IBM que implementa técnicas de resolução de
programação inteira. Todos os experimentos computacionais foram feitos em um
notebook com processador Pentium T4500 2.3 Ghz, 2 GBytes de memória RAM (2x1 GB)
e com o sistema operacional operacional Linux Ubuntu 11.04 de 32 bits.

O GRASP foi configurado com 10 iterações completas do GRASP utilizando um
$\alpha$ de 0.5. Esses valores foram obtidos de forma empírica a partir
dos testes que foram feitos no decorrer do trabalho. A busca local é exaustiva
e finaliza quando não consegue melhorar o valor objetivo da solução,
a maior utilização do poder de processamento do CPU é gasto executando essa
etapa, cerca de 98\% do tempo.

%pode-se colocar uma tabela comparando os valores obtidos para diferentes tipos
% de alpha. Para assim justificar essa escolha. Essa comparação pode ser feita
% com um grafico de convergência em relação a iteração do GRASP (0.25 - 0.5 -
% 0.75) p/ 1 ou 2 instancias

A demonstração da eficiência do algoritimo foi feita baseado na solução ótima
dos problemas que foram obtidas a partir da utilização do modelo
matemático descrito no Capítulo \ref{cap:metodoprop} nas instâncias. A única
instância em que isso não foi possível foi na da TAM estendida onde em um
período de 36 horas (115200 segundos) não foi possível obter nenhuma solução
inteira para o problema. Essa foi a maior instância utilizada no trabalho.

Nesse capítulo $s*$ indica o valor ótimo de uma solução, $s_{m}$ indica a média
dos valores obtidos com todas as execuções do algoritmo, $t_{s*}$ é o
tempo de execução do solver e $t_{s}$ representa o tempo médio de execução do
algoritmo. Por fim $\Delta \%$ (GAP) representa a diferencia em percentual da
média dos valores das soluções obtidas em relação ao valor ótimo da instância,
o seu cálculo é feito com a forma abaixo:

\[  \Delta \% (GAP) = (s - s*)/s* \]

Para os testes foram utilizados duas instâncias diárias, uma da Rio-Sul com 107
voos e outra da TAM com 241 voos. A instância da Rio-Sul foi obtida a partir do
trabalho de \cite{pontes2002} e a da TAM foi obtida através da seleção manual
dos voos através do $site$ da companhia aérea. Com o desenvolvimento do trabalho
essas instâncias passaram a ser resolvidas facilmente pelo sistema. Com a
finalidade de gerar instâncias mais difíceis foi proposto a extensão da
frequência dos voos da instância Rio-Sul e da TAM para uma semana, dessa forma
foi gerado uma instância com 749 voos e outra de 1687. 

Por causa da grande dificuldade de comunicação com as companhias aéreas foi
utilizado um tempo de solo fictício de 20 minutos para todos os aeroportos esse
valor foi escolhido para menter a compatibilidade dos resultados com o trabalho
de \cite{pontes2002}

A resolução dessas instâncias foram parametrizadas levando em consideração dois
cenários. O cenário 1 faz o sequenciamento dos voos sem a permissão de utilizar
nenhum atraso, essa representação é comum nas companhias que não aceitam a
modificação do planejamento inicial. O cenário 2 se utiliza de atrasos
permitindo assim uma maior liberdade na hora da montagem dos trilhos obtendo
assim um melhor aproveitamento da utilização das aeronáves. Os parâmetros
utilizados são detalhados na Tabela \ref{tab:params}.


\begin{table}
\caption{Parametrização dos cenários}\label{tab:params}
\begin{center}


\begin{tabular}{l|rr}
\hline

 & Cenário 1 & Cenário 2 \\
 \hline
 Atraso Maximo & 0 & 10 \\
 Prob. Arc. Tipo 1 & 0.92 & 0.69\\ 
 Prob. Arc. Tipo 2 & 0 & 0.16\\
 Prob. Arc. Tipo 3 & 0.08 & 0.04 \\
 Prob. Arc. Tipo 4 & 0 & 0.01 \\
  
\hline

\end{tabular}
\end{center}
\end{table}




O algoritmo proposto foi executado 100 vezes e apenas a média dos valores da
solução e do tempo foram levados em consideração. As Tabelas \ref{tab:cenario1}
e \ref{tab:cenario2} mostram os resultados obtidos. Para instâncias pequenas a
utilização do solver é suficiente porém percebeu-se que com o aumento do tamanho
da instância o solver leva muito tempo para resolver e uma estratégia híbrida
pode ser o caminho para obter boas soluções em um curto espaço de tempo.

\begin{table}[ht]
\caption{Resultados do cenário 1}\label{tab:cenario1}


\begin{tabular}{l r r r r r}
\hline

Instância 			& $s*$ (trilhos) & $t_{s*}(s)$ & $s_{m}$ (trilhos) & $t_{s}(s)$ &
$\Delta\%$
\\
\hline

Rio Sul 			& 17.138 (17) & 4 s		 	& 17.138 (17) 	& 7 s 		 & 0\\
TAM     			& 35.334 (34) & 26 s	 	& 35.334 (34)	& 36 s 		 & 0\\
Rio Sul Estendida 	& 18.392 (17) & 24192 s	 	& 18.392 (17)	& 64 s 		 & 0\\
TAM Estendida 		& - 		  & 115200 s 	& 49.857 (35)	& 154 s		 & -\\

\hline
\end{tabular}

\end{table}


\begin{table}[ht]
\caption{Resultados do cenário 2}\label{tab:cenario2}


\begin{tabular}{l r r r r r}
\hline

Instância 			& $s*$ (trilhos) & $t_{s*}(s)$ & $s_{m}$ (trilhos) & $t_{s}(s)$ &
$\Delta\%$
\\
\hline

Rio Sul 			& 16.158 (16) & 4 s		 	& 16.158 (16) 	& 7 s 		 & 0\\
TAM     			& 35.015 (34) & 27 s	 	& 35.015 (34)	& 36 s 		 & 0\\
Rio Sul Est. 	& 17.433 (16) & 33001 s	 	& 17.532 (16)	& 65 s 		 &$<$0.01\\ 
TAM Estendida 		& - 		  & 115200 s 	& 48.803 (35)	& 159 s		 & -\\

\hline
\end{tabular}

\end{table}


Pode-se observar que nos dois cenários a solução ótima foi obtida para a
instância da Rio-Sul. Na instância da TAM a solução ótima foi encontrada,
porém, na média o cenário 1 encontrou uma solução bem próxima. Essas duas
instâncias representam um horizonte de tempo de um dia. Na instância da Rio-Sul
estendida que representam uma semana de operação as soluções ficaram em média
0.19 do ótimo para o cenário 1 e 0.18 no cenário 2 com um tempo aproximado de 8
minutos. Para o procedimento programação linear não foram inseridos limites
previamente calculados, de modo que a ferramenta utilizou apenas a relaxação
linear.
  
Alguns ajustes ainda podem melhorar o modelo híbrido para que ele possa se
aproximar mais da solução ótima. A modificação da estrutura a ser otimizada na
busca local pode ser um ponto que ajude a melhorar os resultados, pois a
literatura mostra que esse é um dos pontos mais importantes de uma heurística
híbrida.
 
Uma das grandes dificuldades encontradas no trabalho foi a falta de instâncias
na literatura tornando difícil a comparação de resultados com outras
abordagens. Com isso existe a necessidade de geração de um conjunto de
instâncias e a sua publicação para fins comparativos.
 
Existe ainda a possibilidade de uma implementação paralela que ainda está em
fase de planejamento e que se demonstrar resultados interessantes em tempo
hábil será adicionada a dissertação.
 

%Com a eficiência obtida com o método exato existe uma necessidade de geração de instâncias maiores que possam ser utilizadas parar ajustar e justificar a utilização de uma abordagem mais complexa como o uso de uma metaheurística híbrida.

%Atualmente o método híbrido conseguiu resolver a instância \textit{TAM Estendida} com o tempo de 60s e Custo total de 43344, que parece %ser uma boa solução tendo como base os resultados obtidos com a instância que lhe serviu de base.

%Atualmente existe a necessidade de um melhor ajuste no método híbrido para que ele possa conseguir resultados mais robustos.

%A utilização de uma abordagem exata em conjunto com metaheurísticas está sendo cada vez mais utilizado na literatura. Porém a escolha da estrutura a ser otimizada deve ser bem escolhida para não aumentar demasiadamente a capacidade computacional necessária para resolver o problema.



%Para demonstrar a eficiência em termos de qualidade da solução da metaheurística GILS, realizamos comparações com um  procedimento exato B&B (XPRESS MP, 2004), implementando o modelo STSP apresentado por Lee (1996). Para cada instância da Tabela 1 temos as duas primeiras colunas representando as dimensões das instâncias testadas e as colunas restantes divididas em dois grupos: procedimento B&B e GILS. No caso do procedimento B&B, a coluna z* indica o valor ótimo e a coluna tempo indica o tempo computacional, em segundos, gasto na resolução da instância. Já para o grupo da metaheurística GILS as colunas adicionais, além da coluna tempo, são: iter que indica a iteração onde foi encontrada melhor solução, z que indica o valor obtido pelo GILS e, Δ (gap) que indica a diferença percentual entre as soluções:
%Δ = [ (z – z*) / z* ] x 100.

