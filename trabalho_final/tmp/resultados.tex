  \chapter{Resultados}

%	Para se obter um limite inferior dessas instâncias foi feita uma verificação
%com o algoritmo do Anexo X que permite checar a %quantidade mínima de voos que
%colidem em uma determinada janela de tempo que é definida pelo atraso máximo
%permitido. (Pode-se fazer uma %formula para explicar esse funcionamento). Essa
%quantidade é dito como sendo o limite inferior da instância e é garantido que
%não existe %solução com uma melhor quantidade de trilhos que essa sem que
%nenhum vôo seja excluído.

Neste capítulo apresenta-se os resultados que foram encontrados através da
aplicação do solver, da metaheurística pura e da metaheurística híbrida em um
conjunto de quatro instâncias obtidas ou adaptadas de duas grandes companhias
aéreas, a Rio-Sul e a TAM. Os resultados obtidos com a programação linear
inteira, que foi utilizado como limite inferior - \textit{lower bound} das
soluções. Serão mostrados os resultados da metaheurística pura para mostrar as
vantagens que foram obtidas com a utilização de uma abordagem híbrida.

Na seção \ref{result:ambiente} é apresentado o ambiente de testes as
características dos problemas de testes que foram utilizados nos
experimentos. Nas seções \ref{result:otimizador}, \ref{result:metaheuristica},
\ref{result:hibrid} será mostrado respectivamente os resultados obtidos com o
otimizador, metaheurística pura e metaheurística híbrida. Na seção
\ref{result:finalizacao} são dadas as considerações finais.

\section{Ambiente de testes} \label{result:ambiente}

Todos os algoritmos descritos foram desenvolvidas na linguagem C++ usando o
solver CPLEX Academic 12 da IBM que implementa técnicas de resolução de
programação inteira, os experimentos computacionais foram feitos em um
notebook com processador Pentium T4500 2.3 Ghz, 2 GBytes de memória RAM (2x1 GB)
e com o sistema operacional operacional Linux Ubuntu 11.04 de 32 bits.

As instâncias obtidas tem a duração de um dia, a da Rio-Sul tem 107
voos e a da TAM tem 241 voos. A instância da Rio-Sul foi obtida a partir do
trabalho de \cite{pontes2002} e a da TAM foi obtida através da seleção manual dos voos
pelo $site$ da companhia. Com o desenvolvimento do trabalho essas
instâncias passaram a ser resolvidas facilmente pelo algoritmo. Com a finalidade
de gerar instâncias mais difíceis foi proposto a extensão da frequência dos
voos da instância Rio-Sul e da TAM para uma semana, dessa forma foi gerado uma
instância com 749 voos e outra de 1687. A instância diária da Rio-Sul era
operada por 20 aeronaves, com uma malha que construída manualmente por um
funcionário com larga experiência nesse tipo de serviço.

Para manter a compatibilidade dos resultados com o trabalho de
\cite{pontes2002} foi utilizado o tempo de solo de 20 minutos para todos os
aeroportos. Vale lembrar que esse tempo pode variar de acordo com o aeroporto e
o tipo da frota. Houveram dificudades na obtenção de dados mais detalhados junto
com as companhias aéreas.

O GRASP foi configurado com 1000 iterações na sua aplicação pura e com 10 na
sua aplicação híbrida foi utilizado um $\alpha$ de 0.4. Esses valores foram obtidos de
forma empírica a partir dos testes que foram feitos no decorrer do trabalho. A
busca local é exaustiva e finaliza quando não consegue melhorar o valor
objetivo da solução. O tempo máximo de execução dos algoritmos foram de 32 horas
(115200 segundos).

No método híbrido a maior utilização do poder de processamento do CPU é
gasto executando a etapa de busca local, cerca de 98\% do tempo. 

A demonstração da eficiência do algoritimo foi feita baseado-se na solução ótima
dos problemas que foi obtida a partir da utilização do modelo
matemático descrito no Capítulo \ref{cap:metodoprop}. A única
instância em que isso não foi possível foi na TAM estendida onde no período
máximo estabelecido não foi possível obter nenhuma solução inteira para o
problema. Essa foi a maior instância utilizada no trabalho.

Nesse capítulo $s*$ indica o valor ótimo de uma solução, $s_{m}$ indica a média
dos valores obtidos com todas as execuções do algoritmo, $t_{s*}$ é o
tempo de execução do solver e $t_{s}$ representa o tempo médio de execução do
algoritmo. Por fim $\Delta$ (GAP) representa a diferencia da média dos valores
das soluções obtidas em relação ao valor ótimo da instância, o seu cálculo é
feito com a fórmula abaixo:

\[  \Delta (GAP) = (s - s*)/s* \]

A resolução dessas instâncias foram parametrizadas levando em consideração dois
cenários. O cenário 1 faz o sequenciamento dos voos sem a permissão de utilizar
nenhum atraso, essa representação é comum nas companhias que não aceitam a
modificação do planejamento inicial. O cenário 2 se utiliza de atrasos, de no
máximo 10 minutos, permitindo assim uma maior liberdade na hora da montagem dos
trilhos obtendo assim um melhor aproveitamento da utilização das aeronáves. Os parâmetros
utilizados são detalhados na Tabela \ref{tab:params}.

Os resultados que utilizaram da metaheurística foram obtidos a partir de 100
execuções e somente os valores das médias foram levados em consideração.


\begin{table}
\caption{Parametrização dos cenários}\label{tab:params}
\begin{center}


\begin{tabular}{l|rr}
\hline

 & Cenário 1 & Cenário 2 \\
 \hline
 Atraso Maximo & 0 & 10 \\
 Prob. Arc. Tipo 1 & 0.92 & 0.69\\ 
 Prob. Arc. Tipo 2 & 0 & 0.16\\
 Prob. Arc. Tipo 3 & 0.08 & 0.04 \\
 Prob. Arc. Tipo 4 & 0 & 0.01 \\
  
\hline

\end{tabular}
\end{center}
\end{table}

\section{Resultados do otimizador} \label{result:otimizador}

\begin{table}[ht]
\caption{Resultados do otimizador no cenário 1}\label{tab:otimizadorcenario1}
\begin{center}

\begin{tabular}{l |r r}
\hline

Instância 			& $s*$ (trilhos) & $t_{s*}(s)$
\\
\hline

Rio Sul 			& 17.138 (17) & 4 s		 	\\
TAM     			& 35.334 (34) & 26 s	 	\\
Rio Sul Estendida 	& 18.392 (17) & 24192 s	 	\\
TAM Estendida 		& - 		  & 115200 s 	\\

\hline
\end{tabular}
\end{center}
\end{table}


\begin{table}[ht]
\caption{Resultados do otimizador no cenário 2}\label{tab:cenario2}

\begin{center}
\begin{tabular}{l |r r}
\hline

Instância 			& $s*$ (trilhos) & $t_{s*}(s)$ 
\\
\hline

Rio Sul 			& 16.158 (16) & 4 s		\\
TAM     			& 35.015 (34) & 27 s	\\
Rio Sul Est. 	& 17.433 (16) & 33001 s	 	\\ 
TAM Estendida 		& - 		  & 115200 s\\

\hline
\end{tabular}
\end{center}

\end{table}

Após muitas melhorias na modelagem matemática que foi utilizada no
\textit{solver}, consegiu-se resultados bastante expressivos na resolução das
instâncias. Percebe-se que em um curto espaço de tempo o \textit{solver}
conseguiu resolver de forma eficiente problemas diários de até 241 voos (TAM).
Porém percebeu-se também que esse desempenho diminui bastante com o aumento do
tamanho das instâncias, demorando 6 a 9 horas para resolver uma instância de
749 voos (Rio-Sul estendida) e não conseguindo resolver em 32 horas uma
instância com 1687 voos (TAM estendida) nos cenários que foram modelados.

Percebeu-se também que existe um sequenciamento possível nos voos que permite
economizar até 4 aeronáves para atender a instância da Rio-Sul. Na prática essa
instância era operada com 20 aeronaves.

% O gráfico abaixo permite fazer uma estimativa de quanto tempo 

\section{Resultados da metaheurística pura} \label{result:metaheuristica}

\begin{table}[ht]
\caption{Resultados da metaheurística pura no cenário 1}\label{tab:puracenario1}

\begin{center}
\begin{tabular}{l |r r r}
\hline

Instância 		& $s_{m}$ (trilhos) & $t_{s}(s)$ & $\Delta$
\\
\hline

Rio Sul 			& 19.049 (19) 	& 247 s  & 0.11\\
TAM     			& 42.328 (42)	& 1710 s & 0.20\\
Rio Sul Estendida 	& 21.516 (20)	& 1687 s & 0.17\\
TAM Estendida 		& 51.903 (44)	& xxxx s & -\\

\hline
\end{tabular}
\end{center}
\end{table}


\begin{table}[ht]
\caption{Resultados da metaheurística pura no cenário 2}\label{tab:puracenario2}

\begin{center}
\begin{tabular}{l |r r r}
\hline

Instância 			& $s_{m}$ (trilhos) & $t_{s}(s)$ & $\Delta$
\\
\hline

Rio Sul 			& 18.256 (18) 	& 224 s  & 0.13\\
TAM     			& 41.719 (41)	& 2066 s & 0.19\\
Rio Sul Est. 		& 20.080 (19)	& 1592 s & 0.15\\ 
TAM Estendida 		& 51.032 (44)	& xxxx s & -\\

\hline
\end{tabular}
\end{center}

\end{table}

O algoritmo usando a metaheurística pura teve dificuldades para resolver as
instâncias, inclusive não obteve o valor ótimo de nenhuma delas ficando entre
13\% e 23\% da solução ótima obtida pelo \textit{solver}. Apesar dos
resultados não terem sido o esperado, nota-se que ele ainda é melhor que o que
era utilizado em produção pela Rio-Sul. O GRASP foi configurado para fazer 1000
repetições, pois a quantidade de soluções geradas nesse método é muito
diversificado sendo necessário mais tempo para obter soluções melhores.

Esse comportamento nos levou a acrescentar uma etapa de programação linear
inteira na parte de busca local para obter melhores resultados.


\section{Resultados da metaheurística híbrida} \label{result:hibrid}

\begin{table}[ht]
\caption{Resultados da metaheurística híbrida no cenário
1}\label{tab:hibrydcenario1}

\begin{center}
\begin{tabular}{l |r r r}
\hline

Instância 		& $s_{m}$ (trilhos) & $t_{s}(s)$ & $\Delta$
\\
\hline

Rio Sul 			& 17.138 (17) 	& 7 s 		 & 0\\
TAM     			& 35.334 (34)	& 36 s 		 & 0\\
Rio Sul Estendida 	& 18.392 (17)	& 64 s 		 & 0\\
TAM Estendida 		& 49.857 (35)	& 154 s		 & -\\

\hline
\end{tabular}
\end{center}
\end{table}


\begin{table}[ht]
\caption{Resultados da metaheurística híbrida no cenário
2}\label{tab:hybridcenario2}

\begin{center}
\begin{tabular}{l |r r r}
\hline

Instância 			& $s_{m}$ (trilhos) & $t_{s}(s)$ & $\Delta$
\\
\hline

Rio Sul 			& 16.158 (16) 	& 7 s 		 & 0\\
TAM     			& 35.015 (34)	& 36 s 		 & 0\\
Rio Sul Est. 		& 17.532 (16)	& 65 s 		 &$<$0.01\\ 
TAM Estendida 		& 48.803 (35)	& 159 s		 & -\\

\hline
\end{tabular}
\end{center}

\end{table}


O algoritmo híbrido conseguiu ser consistente na obtenção de bons resultados.
Foram utilizadas 10 iterações do algoritmo, sendo possível obter uma
convergência de 100\% nos resultados mostrados nas tabelas \ref{tab:hibrydcenario1} e
\ref{tab:hybridcenario2}. 

\section{Considerações finais} \label{result:finalizacao}





%pode-se colocar uma tabela comparando os valores obtidos para diferentes tipos
% de alpha. Para assim justificar essa escolha. Essa comparação pode ser feita
% com um grafico de convergência em relação a iteração do GRASP (0.25 - 0.5 -
% 0.75) p/ 1 ou 2 instancias
Para instâncias pequenas a utilização do \textit{solver} se mostrou suficiente,
porém percebeu-se que com o aumento do tamanho da instância o solver leva muito
tempo para resolver e uma estratégia híbrida pode ser o caminho para obter boas
soluções em um curto espaço de tempo.

Pode-se observar que nos dois cenários a solução ótima foi obtida para as
instâncias da Rio-Sul e no cenário 1 para versão diária da TAM. A solução obtida
na TAM Estendida pode ser considerada boa, pois levando em consideração o resultado
ótimo da sua versão diária o resultado ótimo não poderia ter menos que 34
trilhos.
  
Alguns ajustes ainda podem melhorar o modelo híbrido para que ele possa se
aproximar mais da solução ótima. A modificação da estrutura a ser otimizada na
busca local pode ser um ponto que ajude a melhorar os resultados, pois a
literatura mostra que esse é um dos pontos mais importantes de uma heurística
híbrida.
 
Uma das grandes dificuldades encontradas no trabalho foi a falta de instâncias
tornando difícil a comparação de resultados com outras algoritmos. Esse trabalho
disponibiliza as instâncias trabalhadas nos anexos. A instância da Rio-Sul se
encontra nos Anexos \ref{anx:netriosul} e \ref{anx:timeriosul} com o melhor
resultado obtido detalhado no Anexo \ref{anx:resultriosul}. A instância da TAM
se encontra nos Anexos \ref{anx:nettam} e \ref{anx:timetam} com o melhor
resultado obtido detalhado no Anexo \ref{anx:resulttam}. 
 
Experimentos realizados com a implementação paralela discutida no Capítulo
\ref{cap:metodoprop} demonstrou potencial de melhorar os resultados obtidos pois
as soluções iniciais geradas por ela se mostraram melhores que as obtidas
sequencialmente.
 

%Com a eficiência obtida com o método exato existe uma necessidade de geração de instâncias maiores que possam ser utilizadas parar ajustar e justificar a utilização de uma abordagem mais complexa como o uso de uma metaheurística híbrida.

%Atualmente o método híbrido conseguiu resolver a instância \textit{TAM Estendida} com o tempo de 60s e Custo total de 43344, que parece %ser uma boa solução tendo como base os resultados obtidos com a instância que lhe serviu de base.

%Atualmente existe a necessidade de um melhor ajuste no método híbrido para que ele possa conseguir resultados mais robustos.

%A utilização de uma abordagem exata em conjunto com metaheurísticas está sendo cada vez mais utilizado na literatura. Porém a escolha da estrutura a ser otimizada deve ser bem escolhida para não aumentar demasiadamente a capacidade computacional necessária para resolver o problema.



%Para demonstrar a eficiência em termos de qualidade da solução da metaheurística GILS, realizamos comparações com um  procedimento exato B&B (XPRESS MP, 2004), implementando o modelo STSP apresentado por Lee (1996). Para cada instância da Tabela 1 temos as duas primeiras colunas representando as dimensões das instâncias testadas e as colunas restantes divididas em dois grupos: procedimento B&B e GILS. No caso do procedimento B&B, a coluna z* indica o valor ótimo e a coluna tempo indica o tempo computacional, em segundos, gasto na resolução da instância. Já para o grupo da metaheurística GILS as colunas adicionais, além da coluna tempo, são: iter que indica a iteração onde foi encontrada melhor solução, z que indica o valor obtido pelo GILS e, Δ (gap) que indica a diferença percentual entre as soluções:
%Δ = [ (z – z*) / z* ] x 100.

%\begin{table}[ht]
%\caption{Resultados do cenário 1}\label{tab:cenario1}


%\begin{tabular}{l r r r r r}
%\hline

%Instância 			& $s*$ (trilhos) & $t_{s*}(s)$ & $s_{m}$ (trilhos) & $t_{s}(s)$ &
%$\Delta\%$
%\\
%\hline

%Rio Sul 			& 17.138 (17) & 4 s		 	& 17.138 (17) 	& 7 s 		 & 0\\
%TAM     			& 35.334 (34) & 26 s	 	& 35.334 (34)	& 36 s 		 & 0\\
%Rio Sul Estendida 	& 18.392 (17) & 24192 s	 	& 18.392 (17)	& 64 s 		 & 0\\
%TAM Estendida 		& - 		  & 115200 s 	& 49.857 (35)	& 154 s		 & -\\

%\hline
%\end{tabular}

%\end{table}


%\begin{table}[ht]
%\caption{Resultados do cenário 2}\label{tab:cenario2}


%\begin{tabular}{l r r r r r}
%\hline

%Instância 			& $s*$ (trilhos) & $t_{s*}(s)$ & $s_{m}$ (trilhos) & $t_{s}(s)$ &
%$\Delta\%$
%\\
%\hline

%Rio Sul 			& 16.158 (16) & 4 s		 	& 16.158 (16) 	& 7 s 		 & 0\\
%TAM     			& 35.015 (34) & 27 s	 	& 35.015 (34)	& 36 s 		 & 0\\
%Rio Sul Est. 	& 17.433 (16) & 33001 s	 	& 17.532 (16)	& 65 s 		 &$<$0.01\\ 
%TAM Estendida 		& - 		  & 115200 s 	& 48.803 (35)	& 159 s		 & -\\

%\hline
%\end{tabular}

%\end{table}

