  \chapter{Descrição do Software}\label{cap:modelomat}
  
  Durante o desenvolvimento do algoritmo proposto existiu a necessidade de
  vizualizar as soluções que iam sendo criadas com a finalidade de observar
  caractéristicas que pudessem ser modificadas para melhorar a qualidade da
  solução, adicionalmente viu-se a necessidade de que essa interface fosse boa o
  suficiente para ser utilizada comercialmente.
  
  A interface foi construída utilizando a linguagem JAVA e integrando-se com o
  algoritmo em C++ através da chamada do executável do algoritmo através da
  escrita dos dados necessáros para a sua execução em um arquivo e pela
  posterior leitura da resposta em outro arquivo.
  
  As funcionalidades implementadas permitiram a vizualização da solucão, com
  estatistícas como número de trilhos, atraso máximo, total de atraso, número de
  reposicionamento e custo da solução. Na figura X pode-se ver uma solução que
  foi gerada nela a cor laranja indica o tempo de solo e a outra parte o tempo
  de voo. Caso o voo seja azul ele não possui atrasos, caso ele seja vermelho
  ele possui atrasos e caso ele seja rosa ele é um voo de reposicionamento.
  
  
