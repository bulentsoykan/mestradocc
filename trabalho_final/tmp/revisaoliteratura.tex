\chapter{Revisão da literatura}
  
	
O trabalho de Argüello e Bard \cite{arguelo1997}  apresenta um
algoritmo baseado no GRASP para resconstruir trilhos de aeronaves que tenham
sofrido atrasos durante o decorrer do dia e tem como principal objetivo a
reducão dos custos da reatribuição das aeronaves aos voos que é mensurado
apartir do atraso dado aos voos e pelo número de voos cancelados. 

Nesse trabalho foi utilizado a ideia de trilhos cancelado, que é formado
pelos voos que não serão levados em consideração na solução final. 
		
A reconstrução dos trilhos é feita com a utilização sucessiva de três
estruturas de vizinhança, \textit{flight routing augmentation}, \textit{partial route exchange} e
\textit{simple circuit cancelation} onde as duas primeiras são aplicadas em um
par de trilhos e a terceira é aplicada em trilhos individualmente

O \textit{flight routing augmentation} remove uma sequência de voos do
primeiro trilho e acrescenta eles no trilho de destino, ou seja, o segundo
trilho é acrescido dos voos que foram removidos do primeiro. O trilho de
destino pode crescer de três formas. Primeiro um circuito pode ser inserida
no seu início. Um circuito é uma sequência de voos que se origina e termina no
mesmo aeroporto. A segunda forma é a adição de um circuito em algum lugar
entre o primeiro e o ultimo voo. A terceira forma envolve a adição de uma
sequência de voos, que não precisa ter a mesma origem e destino, e a sua
inserção no final do segundo trilho. Lembrando que apenas movimentos viáveis
são avaliados. 
		
O movimento de \textit{partial route exchange} é uma simples
troca de um par de sequências de voos. Dois tipos de trocas são possíveis. A
primeira é a troca de duas sequências que possuam os mesmo extremos. E a
segunda é uma troca que resulta na mudança do aeroporto de destino. Um trilho
de cancelamento não pode trocar seus aeroportos de destino com outro trilho
pois esse movimento poderia causar uma violação na restrição de balanceamento
de aeronáves.
		
O \textit{simple circuit cancelation} é feito em um único trilho e ela
simplesmente remove um circuito desse trilho e efetua a criação de um novo
trilho de cancelamento. Além disso foi desenvolvido um modelo matemático que
foi utilizado apenas para a obtenção de um limite inferior (\textit{lower
bound}).
		
		
Mercier e Soumis \cite{mercier2007} resolveram o PCTA em conjunto com
o problema de escala de tripulantes pois Cordeau et al. \cite{cordeau2001},
Klabjan et al. \cite{klabjan2002} e Cohn e Barnhart \cite{mainville2003}
mostraram que a resolução desses problema de forma integrada pode gerar
soluções que são significantemente melhor que as geradas de forma sequencial.
Com essa finalidade eles proporam uma formulação compacta do problema e
utilizaram o método de decomposição de Benders com um procedimento de geração
de restrição dinâmica para resolve-lo. Com a agregação desses dois problemas a
resolução se tornou pesada e viável apenas para instâncias diárias. Os testes
do algoritmo foram baseados em instâncias contendo no máximo 500 voos que
foram fornecidas por duas grandes companhias aereas, porém elas não se
encontram disponíveis no artigo. 
		
Pontes R., et al \cite{pontes2002} utilizaram a fase de construção do GRASP
para resolver o PCTA, também propuseram um modelo matemático que foi
adaptado para auxiliar na geração da nossa solução. Além disso uma instância
da Rio-Sul foi disponibilizada para a realização de testes. Com o solver eles
conseguiram obter a solução ótima dessa instância mas o autor informou que
essa resolução demorou dias para finalizar. Com a utilização da heurística
eles conseguiram apenas se aproximar dessa solução porém com um tempo de 384
segundos.
		
Em \cite{mohamed2011} Mohamed et al. resolveu de forma integrada o problema
de atribuição de frota e o problema de construção de trilhos de aeronaves,
para uma pequena empresa de aviação a TunisAir. Além disso as restrições de
manutenção não foram levadas em consideração pelo fato dela poder ser feita
em todos os aeroportos em que as aeronaves passam a noite.
		
%GRASPs have been used to find high quality solutions to a variety of logistics and combi- natorial optimization problems including maintenance base %planning (Feo and Bard, 1989), machine scheduling (Feo et al., 1991), and number partitioning (Argu ̈ello et al., 1996) to name a few.
%ão
 